%\documentclass[journal]{IEEEtran}
\documentclass{article}

% Packages
\usepackage{cite}
\usepackage{graphicx}
\usepackage{subfigure}
\usepackage{float}
\usepackage{url}
\usepackage{color}
\usepackage{amsmath}


\begin{document}

% paper title
% can use linebreaks \\ within to get better formatting as desired
\title{Probabilistic~Robotics~Exercise~1}
%

\author{Rodrigo~Caye~Daudt}





% make the title area
\maketitle



%%%%%%%%%%%%%%%%%%%%%%%%%%%%%%%%%%%%%%%%%%%%%%%%%%%%%%%%%%%%%%%%%%%%%%%%%%%%%%%
\section{Problem Description}

A robot manipulator is equipped with a tactile sensor in its gripper. When the robot is commanded to grasp an object, if the object was not already grasped it succeeds with 0.7 probabilities but fails with a 0.3 probability, otherwise it continues grasping the object with 0.9 probability but can fail with 0.1 probability.
Whenever an object is hold within the gripper, the tactile sensor detects the situation with a 0.6 probability but fails in the detection with a 0.4 probability.
If the gripper is free, the tactile sensor detects the situation with 0.8 probability but fails with 0.2 probability.

The following sequence of actions/sensor readings have been executed:

1) The action GRASP is executed

2) The tactile sensor detects NO OBJECT GRASPED.

%%%%%%%%%%%%%%%%%%%%%%%%%%%%%%%%%%%%%%%%%%%%%%%%%%%%%%%%%%%%%%%%%%%%%%%%%%%%%%%
\section{Solution}

From the problem description we can organize the information as follows:

\textbf{State transition probabilities:}

\[
p(X_t=ObjectGrasped|U_t=Grasp, X_{t-1}=NoObjectGrasped) = 0.7
\]
\[
p(X_t=NoObjectGrasped|U_t=Grasp, X_{t-1}=NoObjectGrasped) = 0.3
\]
\[
p(X_t=ObjectGrasped|U_t=Grasp, X_{t-1}=ObjectGrasped) = 0.9
\]
\[
p(X_t=NoObjectGrasped|U_t=Grasp, X_{t-1}=ObjectGrasped) = 0.1
\]


\textbf{Measurement probabilities:}

\[
p(Z_t=ObjectGrasped|X_t=ObjectGrasped) = 0.6
\]

\[
p(Z_t=NoObjectGrasped|X_t=ObjectGrasped) = 0.4
\]

\[
p(Z_t=ObjectGrasped|X_t=NoObjectGrasped) = 0.2
\]

\[
p(Z_t=NoObjectGrasped|X_t=NoObjectGrasped) = 0.8
\]

\textbf{Initial belief:}

\[
bel(X_0=ObjectGrasped) = 0.5
\]

\[
bel(X_0=NoObjectGrasped) = 0.5
\]

\subsection{Prediction}

In this step we should apply the prediction step of the Bayes filter to calculate the new belief predictions, given that $U_1=Grasp$.


\begin{align*}
\overline{bel}(&X_1=ObjectGrasped) = bel(X_0=ObjectGrasped) \\
			&\cdot p(X_t=ObjectGrasped|U_t=Grasp, X_{t-1}=ObjectGrasped) \\
			&+ bel(X_0=NoObjectGrasped) \\
			&\cdot p(X_t=ObjectGrasped|U_t=Grasp, X_{t-1}=NoObjectGrasped)\\
			&=0.5 \cdot 0.9 + 0.5 \cdot 0.7 = 0.8
\end{align*}

\begin{align*}
\overline{bel}(&X_1=NoObjectGrasped) = bel(X_0=ObjectGrasped) \\
			&\cdot p(X_t=NoObjectGrasped|U_t=Grasp, X_{t-1}=ObjectGrasped) \\
			&+ bel(X_0=NoObjectGrasped) \\
			&\cdot p(X_t=NoObjectGrasped|U_t=Grasp, X_{t-1}=NoObjectGrasped)\\
			&=0.5 \cdot 0.1 + 0.5 \cdot 0.3 = 0.2
\end{align*}


\subsection{Correction}

Then, we apply the correction step of the Bayes filter considering we obtained a measurement $Z_1=NoObjectGrasped$:

\begin{align*}
bel(X_1&=ObjectGrasped) = \\
					&\eta \cdot p(Z_t=NoObjectGrasped|X_t=ObjectGrasped)\\
					& \cdot  \overline{bel}(X_1=ObjectGrasped)\\
					& = \eta \cdot 0.4 \cdot 0.8\\
					& = \eta \cdot 0.32
\end{align*}

\begin{align*}
bel(X_1&=NoObjectGrasped) = \\
					& \eta \cdot p(Z_t=NoObjectGrasped|X_t=NoObjectGrasped)\\
					& \cdot  \overline{bel}(X_1=NoObjectGrasped)\\
					& = \eta \cdot 0.8 \cdot 0.2\\
					& = \eta \cdot 0.16
\end{align*}

We then calculate $\eta$:

\begin{align*}
\eta &= \frac{1}{\frac{bel(X_1=ObjectGrasped)}{\eta} + \frac{bel(X_1=NoObjectGrasped)}{\eta}}\\
	& = \frac{1}{0.32 + 0.16} = \frac{1}{0.48}
\end{align*}

Therefore:

\[
bel(X_1=ObjectGrasped) = \frac{2}{3}
\]


\[
bel(X_1=NoObjectGrasped) = \frac{1}{3}
\]


\subsection{Final Results}

The table can then be filled as follows:

\hspace{5pt}

\begin{tabular}{l | r  r}
Executed & Probability & Probability\\
Action & Object Grasped & No Object Grasped\\
\hline
$U_t = Grasp$ & 0.8 & 0.2 \\
$Z_t = NoObjectGrasped$ & 0.667 & 0.333 
\end{tabular}





%\bibliographystyle{IEEEtran}
%\bibliography{Template_Daudt}


\end{document}


