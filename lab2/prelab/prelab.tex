\documentclass[11pt,oneside,a4paper]{article}

% Packages
\usepackage{cite}
\usepackage{graphicx}
\usepackage{subfigure}
\usepackage{float}
\usepackage{url}

\begin{document}

% paper title
% can use linebreaks \\ within to get better formatting as desired
\title{Probabilistic Robotics Lab 2 \\ Prelab}
%
%
% author names and IEEE memberships
% note positions of commas and nonbreaking spaces ( ~ ) LaTeX will not break
% a structure at a ~ so this keeps an author's name from being broken across
% two lines.
\author{Rodrigo~Caye~Daudt}


% The paper headers
\markboth{Prelab 2}%
{Geometry}


% make the title area
\maketitle




% Note that keywords are not normally used for peerreview papers.
%\begin{IEEEkeywords}
%IEEEtran, journal, \LaTeX, paper, template. ln505
%\end{IEEEkeywords}



%%%%%%%%%%%%%%%%%%%%%%%%%%%%%%%%%%%%%%%%%%%%%%%%%%%%%%%%%%%%%%%%%%%%%%%%%%%%%%%
\section{Line parameters from points}

Given $P_1 = (x_1 , y_1)$ and $P_2 = (x_2 , y_2)$, we can calculate the coefficients $a$, $b$, and $c$ of the equation $$ax + by + c = 0$$ that describes the line that passes through $P_1$ and $P_2$ using Eqs. \ref{a}, \ref{b}, and \ref{c}.

\begin{equation}\label{a}
a = y_1 - y_2
\end{equation}

\begin{equation}\label{b}
b = x_2 - x_1
\end{equation}

\begin{equation}\label{c}
c = x_1 y_2 - x_2 y_1
\end{equation}

%%%%%%%%%%%%%%%%%%%%%%%%%%%%%%%%%%%%%%%%%%%%%%%%%%%%%%%%%%%%%%%%%%%%%%%%%%%%%%%
\section{Distance between point and line}

Given a line defined by $$ax + by + c = 0$$ and a point $P_0 = (x_0 , y_0 )$, the minimum distance between the point and the line is given by Eq. \ref{dist}.

\begin{equation}\label{dist}
distance = \frac{|ax_0 +by_0 +c|}{\sqrt{a^2 + b^2}}
\end{equation}




\end{document}


