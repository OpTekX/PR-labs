\documentclass[journal]{IEEEtran}

% Packages
\usepackage{cite}
\usepackage{graphicx}
\usepackage{subfigure}
\usepackage{float}
\usepackage{url}
\usepackage{color}
\usepackage[]{algorithm2e}
\usepackage{amsmath}

\begin{document}

% paper title
% can use linebreaks \\ within to get better formatting as desired
\title{Probabilistic~Robotics~Prelab~4 \\ Extended~Kalman~Filter}
%

\author{Rodrigo~Caye~Daudt}





% make the title area
\maketitle



%%%%%%%%%%%%%%%%%%%%%%%%%%%%%%%%%%%%%%%%%%%%%%%%%%%%%%%%%%%%%%%%%%%%%%%%%%%%%%%
\section{Transforming From World Coordinates To Robot Coordinates}\label{aaa}

\IEEEPARstart{T}{he} transformations that can be used to obtain $(\rho_r,\phi_r)$ from $(\rho_w,\phi_w)$ given the robot's position $(x_r,y_r,\theta)$ are described below. When programming, the \textit{angle\_wrap} function should be used after the final calculation of $\phi_r$.

\begin{algorithm}
$\rho_r ' = \rho_w - x_r cos\phi_w - y_r sin \phi_w$\;
$\phi_r ' = \phi_w - \theta$\;
\eIf{$\rho_r ' >= 0$}{
$\rho_r = \rho_r '$\;
$\phi_r = \phi_r '$\;
}{
$\rho_r = -\rho_r '$\;
$\phi_r = \phi_r ' + \Pi$\;
}
\end{algorithm}



%%%%%%%%%%%%%%%%%%%%%%%%%%%%%%%%%%%%%%%%%%%%%%%%%%%%%%%%%%%%%%%%%%%%%%%%%%%%%%%
\section{Measurement Equations And Jacobian H}

The system's state $x$ is defined as the following vector:

\[
x = 
\begin{bmatrix}
x_r \\
y_r \\
\theta
\end{bmatrix}
\]

And the model we will use for the system dynamics is:

\[
\hat{x_k} = 
\begin{bmatrix}
\hat{x}_{r_{k-1}} + (\Delta x + w_x) \cdot cos \theta_{k-1} - (\Delta y + w_y) \cdot sin \theta_{k-1} \\
\hat{y}_{r_{k-1}} + (\Delta x + w_x) \cdot sin \theta_{k-1} + (\Delta y + w_y) \cdot cos \theta_{k-1} \\
\theta_{k-1} + (\Delta \theta + w_\theta)
\end{bmatrix}
\]

With the model above we can calculate the $A_k$ and $W_k$ matrices to be used for the EKF:

\[
A_k = 
\begin{bmatrix}
1 & 0 & -\Delta x\cdot sin \theta_{k-1} - \Delta y \cdot cos \theta_{k-1} \\
0 & 1 & \Delta x\cdot cos \theta_{k-1} - \Delta y \cdot sin \theta_{k-1} \\
0 & 0 & 1 \\
\end{bmatrix}
\]

\[
W_k = 
\begin{bmatrix}
cos \theta_{k-1} & -sin \theta_{k-1} & 0 \\
sin \theta_{k-1} & cos \theta_{k-1} & 0 \\
0 & 0 & 1 \\
\end{bmatrix}
\]

As was stated in Section \ref{aaa}, our definition of $h(x_k,v_k)$ has to be split in two cases. The case for $\rho_r ' = 0$ since be treated with care since wither case could be valid.

\newpage

If $\rho_r ' >= 0$:

\[
h(x_k,v_k) = 
\begin{bmatrix}
\rho_r + v_\rho \\
\phi_r + v_\phi
\end{bmatrix}
=
\begin{bmatrix}
\rho_w - x_r cos\phi_w - y_r sin \phi_w + v_\rho \\
\phi_w - \theta + v_\phi
\end{bmatrix}
\]

\[
H_k = 
\begin{bmatrix}
-cos\phi_w & -sin \phi_w & 0 \\
0 & 0 & -1
\end{bmatrix}
\]

Else:

\[
h(x_k,v_k) = 
\begin{bmatrix}
\rho_r + v_\rho \\
\phi_r + v_\phi
\end{bmatrix}
=
\begin{bmatrix}
-\rho_w + x_r cos\phi_w + y_r sin \phi_w + v_\rho \\
\phi_w - \theta + \Pi + v_\phi
\end{bmatrix}
\]

\[
H_k = 
\begin{bmatrix}
cos\phi_w & sin \phi_w & 0 \\
0 & 0 & -1
\end{bmatrix}
\]


%\bibliographystyle{IEEEtran}
%\bibliography{Template_Daudt}


\end{document}


